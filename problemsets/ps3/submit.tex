\documentclass[11pt]{article}
\usepackage[margin=1.25in]{geometry}
\usepackage{amsmath, amssymb}
\usepackage{setspace}
\doublespace

\begin{document}
    \begin{enumerate}
        \item Problem PS 3.1 $\rightarrow$ Use induction to prove the following claims
        \begin{enumerate}
            \item $P(n) = L_0 + L_1 + L_2 + ... + L_n = L_{n+2} - 1$
            \begin{itemize}
                \item BASE CASE: $L_0 = L_2 - 1$
                \item $L_0 = 2$, $L_2 = 3$, $2 = 3-1 \rightarrow 2 = 2$, base case is true
                \item INDUCTION: If $L_0 + L_1 + ... + L_n + L_{n+1} = L_{n+1+2} - 1$
                \item LHS: $L_{n+2} - 1 + L_{n+1}$
                \item We know that $L_n = L_{n-1} + L_{n-2}$
                \item Then, $L_{n+3} = L_{n+3-1} + L_{n+3-2} = L_{n+2} + L_{n+1}$
                \item We can substitute the LHS, it becomes $L_{n+3} -1$, which is the same as the RHS.
                \item With induction, we proved that the following statement is \textbf{T} for all n. $\blacksquare$
            \end{itemize}
            \item $L_n = F_{n-1} + F_{n+1}$
            \begin{itemize}
                \item BASE CASE: $L_1 = F_{0} + F_{2}$
                \item $L_1 = 1$, $F_{0} = 0$, $F_{2} = 1$ 
                \item $1 = 1$, base case is true
                \item INDUCTION: $L_{n+1} = F_n + F_{n+2}$
                \item We know that $L_{n+1} = L_n + L_{n-1}$
                \item We can substitute $L_n$ from above, $L_n = F_{n-1} + F_{n+1}$ and $L_{n-1} = F_{n-2} + F_{n}$
                \item With this, we get $F_{n-1} + F_{n+1} + F_{n-2} + F_{n}$
                \item Reorganize to get $F_{n-2} + F_{n-1} + F{n} + F{n+1}$
                \item We can get $F_{n} + F{n+2}$ from this.
                \item With induction, we proved that the following statement is \textbf{T} for all n. $\blacksquare$
            \end{itemize}
        \end{enumerate}
        \item Problem 3.2 $\rightarrow$ Use induction to prove
        \begin{enumerate}
            \item $2 + 6 + 12 + ... + (n^2 - n) = \frac{n(n^2-1)}{3}$
            \begin{itemize}
                \item BASE CASE: for n = 0: $0^2 - 0 = \frac{0(0^2-1)}{3} = 0$, base case is true
                \item INDUCTION: $2 +6 + 12 + ... + (n^2 - n) + ((n+1)^2-(n+1)) = \frac{(n+1)((n+1)^2-1)}{3}$
                \item work with LHS, simplify RHS as necessary, but don't change anything
                \item $\frac{n^3 - n}{3} + (n^2 + n) = \frac{(n+1)(n^2 + 2n)}{3}$
                \item simplify to get $\frac{n^3 -n + 3(n^2+n)}{3} = $ RHS $\frac{n^3 + 3n^2 + 2n}{3}$
                \item simplify futher to get LHS $\frac{n^3 - n + 3n^2 + 3n}{3} = \frac{n^3 + 3n^2 + 2n}{3}$, which is the same as the RHS
                \item With induction, we proved that the following statement is \textbf{T} for all n. $\blacksquare$
            \end{itemize}
        \end{enumerate}
        \item Prove that for $n \geq 1$, there is $k \geq 0$ and $\ell$ odd such that $n = 2^k\ell$
        \begin{itemize}
            \item BASE CASE: $n = 1$ and $k = 0$, therefore plugging into $Q(n) = P(n) = 2^k\ell$ becomes $1 = 2^0\ell$
            \item in this case, $\ell = 1$, which makes it odd, base case passed
            \item INDUCTION STEP: $n = n+1$ and $k = k + 1$
            \item $n + 1 = 2 ^ {k+1} \ell \rightarrow$ plug in $n$ for $2^k\ell$
            \item $2^k\ell + 1 = 2^{k+1}\ell$, knowing that $\ell$ will be odd, we can substitute it for $2w + 1$
            \item $2^k(2w+1) + 1 = 2^{k+1}\ell$
            \item $2^k2w+2^k+1 = 2^k\times2\times\ell$
        \end{itemize}
    \end{enumerate}
\end{document}