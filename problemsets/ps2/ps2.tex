\documentclass[12pt]{article}
\usepackage[margin=1.25in]{geometry}
\usepackage{amsmath, amssymb}
\usepackage{setspace}
\doublespace

\begin{document}
    \begin{enumerate}

        \item assume n is an integer, give direct and contraposition proofs
        \begin{enumerate}
            \item ($n^3 + 5$ is odd) $ \rightarrow $ (n is even)
            \begin{enumerate}
                \item direct proof
                \begin{itemize}
                    \item $n^3 + 5$ is odd, $n^3 + 5 = 2k + 1$, $n^3 = 2k -4$
                    \item $n = \sqrt[3]{2(k-2)}$
                    \item a cube root of an even number is always even by definition, statement claimed in p is true
                \end{itemize}
                \item contraposition proof
                \begin{itemize}
                    \item assume n is not even, n is odd, $n = 2k + 1$
                    \item $n^3 + 5 = (2k+1)^3 + 5 => 8k^3 + 12k^2 + 6k + 6$
                    \item $8k^3 + 12k^2 + 6k + 6 = 2(4k^3 + 6k^2 + 3k + 3)$
                    \item we have shown that p is even when n is odd, the statement claimed is true 
                \end{itemize}
            \end{enumerate}
            \item (3 does not divide $n$) $ \rightarrow $ (3 divides $n^2 + 2$)
            \begin{enumerate}
                \item direct proof
                \begin{itemize}
                    \item if 3 does not divide n, then $n = 3k + 1$
                    \item assuming p is true, 3 divides $n^2 + 2$, so $n^2 + 2 = 3k$
                    \item $n^2 = 3k-2$, then $n = \sqrt{3k-2}$
                    \item plugging in $n = 3k$, we get $n = \sqrt{n-2}$
                    \item this is not true for all cases, therefore, this statement is false
                \end{itemize}
                \item contraposition proof
                \begin{itemize}
                    \item 3 does not divide by $n^2 + 2$
                    \item therefore, $n^2 + 2 \neq 3k$, for some integer k
                    \item 3 does not divide n, $n \neq 3k$
                    \item $(3k)^2 + 2 \neq 3k$
                \end{itemize}
            \end{enumerate}
        \end{enumerate}
        
        \item prove by contradiction
        \begin{enumerate}
            \item $(x, y) \in \mathbb{Z}^2 \rightarrow x^2 - 4y -3 \neq 0$
            \begin{itemize}
                \item assume $x^2 - 4y - 3 = 0$
                \item rearranging variables to isolate x, we get $x^2 = 4y + 3$
                \item we now know that $x^2$ is odd, therefore $x^2 = 2k+1$
                \item substitute $x^2$ in, $2k + 1 - 4y = 3$
                \item isolate y, we get $\frac{2k + 1 - 3}{4} = y$, $y = \frac{2k-2}{4}$
                \item simplify to get $y = \frac{k - 1}{2}$, uh oh, there are integers $k$ that makes $y$ not a positive integer
                \item the statement is true due to proof by contradiction
            \end{itemize}
        \end{enumerate}
        
        \item prove these if and only if, prove two implications
        \begin{enumerate}
            \item prove: 4 divides $n \in \mathbb{Z}$ IF AND ONLY IF $n = 1 + (-1)^k(2k-1)$ for $k \in \mathbb{N}$. (Try $n < 0$, $n = 0$, $n > 0$; $k$ is even/odd.)
            \begin{enumerate}
                \item p $\rightarrow$ q
                \begin{itemize}
                    \item direct proof: $4w = n \rightarrow n = 1 + (-1)^k(2k-1)$ 
                    \item from plugging in, we get $4w = 1 + (-1)^k(2k-1)$
                    \item if $k$ is even, then $k = 2w$, from plugging in, we get $4w = 1 + (2(2w) - 1)$, equal to $4w = 4w$, from above, we stated that $4$ divides $n$, therefore $n = 4w$
                    \item if $k$ is odd, then $k = 2w + 1$, from plugging in, we get $4w = 1 + (-1)(2(2w+1)-1)$, equal to $4w \neq 1-4w$, then $n = 1-4w$, which does not divide 4
                    \item if $k$ = 0, then it becomes $4w = 0$, then $n = 0$ as well, which is divisible by $4$
                    \item plug in 4w for n in the other equation, then do the three cases where k is even or odd or equal to 0
                \end{itemize}
                \item q $\rightarrow$ p
                \begin{itemize}
                    \item direct proof
                    \item assuming that p is true
                    \item assuming $k = 2w$ for even cases, $k = 2w+1$ for odd cases, $k = 0$ for 0 cases
                    \item even cases of k, $n = 1 + (2(2w) - 1) = 4w$, then $n = 4w$, it does divide by 4
                    \item odd cases of k $n = 1 + (-1)(2(2w+1) -1)$, then $n = 1-4w$, it does not divide by 4
                    \item 0 case of k $n = 0$, it is divisible by 4
                \end{itemize}
            \end{enumerate}
        \end{enumerate}
        
        \item determine the type of proof and prove
        \begin{enumerate}
            \item If $n$ is odd, then $n^2 - 1$ is divisible by 8.
            \begin{itemize}
                \item this statement can be proven by a direct proof
                \item if $n$ is odd, then $n = 2k + 1$
                \item then with $n^2 - 1$, you can substitute $n$ for $2k+1$
                \item becomes $(2k+1)^2 - 1 = 4k^2 + 4k + 1 - 1$
                \item at its simplest form, it is $4(k^2 + k) \neq 8k$
                \item by direct proof, the statement is false
            \end{itemize}
        \end{enumerate}
        
    \end{enumerate}
\end{document}