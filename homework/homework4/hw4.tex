\documentclass[11pt]{article}
\usepackage{datetime}
\usepackage{color,array,graphics}
\usepackage{enumerate}
\usepackage[pdftex, colorlinks, linkcolor=red,citecolor=red,urlcolor=blue]{hyperref}
\usepackage{ulem}

\setlength{\parindent}{0cm}

\setlength{\parskip}{0.3cm plus4mm minus3mm}

\textwidth  6.5in
\oddsidemargin +0.0in
\evensidemargin +0.0in
\textheight 9.0in
\topmargin -0.5in

\usepackage{upquote,textcomp}
\usepackage{amssymb,amsmath,amsfonts,amsthm}
\usepackage{graphicx}
\usepackage{multicol}
\usepackage[T1]{fontenc}

\def\OR{\vee}
\def\AND{\wedge}
\def\imp{\rightarrow}

\DeclareSymbolFont{AMSb}{U}{msb}{m}{n}
\DeclareMathSymbol{\N}{\mathbin}{AMSb}{"4E}
\DeclareMathSymbol{\Z}{\mathbin}{AMSb}{"5A}
\DeclareMathSymbol{\R}{\mathbin}{AMSb}{"52}
\DeclareMathSymbol{\Q}{\mathbin}{AMSb}{"51}
\DeclareMathSymbol{\I}{\mathbin}{AMSb}{"49}
\DeclareMathSymbol{\C}{\mathbin}{AMSb}{"43}

\begin{document}
\thispagestyle{empty}   %% skips page number on the first page

\begin{center}
\large
\textbf{CSCI 2200 --- Foundations of Computer Science (FoCS) \\
Homework 4 (document version 1.0)}
\end{center}

\section*{Overview}
\begin{itemize}
\item This homework is due by 11:59PM on Thursday, November~3
\item You may work on this homework in a group of no more than four students;
  unlike recitation problem sets,
  \textbf{your teammates may be in any section}
\item You may use at most \textbf{three} late days on this assignment
\item Please start this homework early and ask questions during
  office hours; % and at your September~14 recitation section;
  also ask (and answer) questions on the Discussion Forum 
\item Please be concise in your answers;
  even if your solution is correct, if it is not well-presented,
  you may still lose points
\item You can type or hand-write (or both) your solutions
  to the required graded problems below;
  \textbf{all work must be organized in one PDF that lists
  all teammate names}
\item You are strongly encouraged to use LaTeX, in particular for
  mathematical symbols;
  see references in Course Materials
\item \textbf{EARNING LATE DAYS:}
  for each homework that you complete using LaTeX
  (including any tables, graphs, etc., i.e.,~no hand-written anything),
  you earn one additional late day;
  you can draw graphs and other diagrams
  in another application and include them as image files
\item To earn a late day, you must submit your LaTeX files
  (i.e.,~\verb+*.tex+) along with your one required PDF file---please name
  the PDF file \verb+hw4.pdf+
\item Also note that the earned late day can be used
  retroactively, even back to the first homework assignment!
\end{itemize}

\vspace{0.2in}

%\begin{center}
%\includegraphics[scale=0.5]{math-bitmoji.png}
%\end{center}

\newpage
\section*{Warm-up exercises}
The problems below are good practice problems to work on.
Do not submit these as part of your homework submission.
\textbf{These are ungraded problems.}

\begin{multicols}{2}
\begin{itemize}

% 10+
\item \textbf{Problem 9.20.}
\item \textbf{Problem 9.23(a).}
\item \textbf{Problem 9.28.}
\item \textbf{Problem 10.8.}
\item \textbf{Problem 10.10.}
\item \textbf{Problem 10.11.}
\item \textbf{Problem 10.15(a-b).}
\item \textbf{Problem 10.22(b-c,e).}
\item \textbf{Problem 10.29.}
\item \textbf{Problem 11.3.}
\item \textbf{Problem 11.5.}
\item \textbf{Problem 11.10.}
\item \textbf{Problem 11.13.}
\item \textbf{Problem 11.15.}

\end{itemize}
\end{multicols}

\section*{Graded problems}
The problems below are required and will be graded.
\begin{itemize}

% 8+
\item \textbf{*Problem 9.23(b).}
\item \textbf{*Problem 10.13.}
\item \textbf{*Problem 10.18(a-b).}
\item \textbf{*Problem 10.22(a,d).}
\item \textbf{*Problem 10.32.}
\item \textbf{*Problem 11.11.}
\item \textbf{*Problem 11.17.}
\item \textbf{*Problem 11.27.}

\end{itemize}

Some of the above problems (graded and ungraded)
are transcribed in the pages that follow.

Graded problems are noted with an asterisk~(*).

If any typos exist below, please use the textbook description.

\newpage
\begin{itemize}

\item \textbf{Problem 9.20.}
Prove or disprove:
\begin{enumerate}[(a)]
\begin{multicols}{4}
\item $\displaystyle\frac{n^3+2n}{n^2+1}\in\Theta(n)$
\item $(n+1)!\in\Theta(n!)$
\item $n^{1/n}\in\Theta(1)$
\item $(n!)^{1/n}\in\Theta(n)$
\end{multicols}
\end{enumerate}

\vspace{0.1in}

\item \textbf{Problem 9.23(a).}
Prove by contradiction: (a)~$n^3\not\in O(n^2)$

\vspace{0.1in}

\item \textbf{*Problem 9.23(b).}
Prove by contradiction: (b)~$2^n\not\in O(3^n)$

\vspace{0.1in}

\item \textbf{Problem 9.28.}
For recurrence $f(0)=1$; $f(n)=nf(n-1)$, compare $f(n)$ with (a)~$2^n$ (b)~$n^n$.

\vspace{0.1in}

\item \textbf{Problem 10.8.}
What natural numbers are relatively prime to 2, 3, and 6?

\vspace{0.1in}

\item \textbf{Problem 10.10.}
For any $m,n,x\in\Z$, prove that $\text{gcd}(m,n)=\text{gcd}(m,n-mx)$.

\vspace{0.1in}

\item \textbf{Problem 10.11.}
Use Euclid's algorithm and the remainders generated to solve these problems.
\begin{enumerate}[(a)]
\item Compute $\text{gcd}(1200,2250)$ and find $x,y\in\Z$
  for which $\text{gcd}(1200,2250)=1200\cdot x+2250\cdot y$.
\item Find $x,y$ as in (a) but with the additional requirement that $x\le 0$ and $y\ge 0$.
\end{enumerate}

\vspace{0.1in}

\item \textbf{*Problem 10.13.}
Let $d=\text{gcd}(m,n)$, where $m,n>0$.
Bezout gives $d=mx+ny$, where $x,y\in\Z$.
Prove or disprove:
\begin{enumerate}[(a)]
\item It is always possible to choose: (i)~$x>0$ (ii)~$x<0$.
\item It is possible to find another $x,y\in\Z$ for which $0<mx+ny<d$.
\item It is always possible to find $a,b\in\Z$ for which $ax+by=1$.
\end{enumerate}

\vspace{0.1in}

\item \textbf{Problem 10.15(a-b).}
Prove.
\begin{enumerate}[(a)]
\item If $a$ divides $bc$ and $\text{gcd}(a,b)=1$ then $a$ divides $c$.
\item For any prime $p$, if $p|a_1a_2\ldots a_n$ then $p$ divides one of the $a_i$.
\end{enumerate}

\vspace{0.1in}

\item \textbf{*Problem 10.18(a-b).}
The Fibonacci numbers are: $F_1=F_2=1$ and $F_n=F_{n-1}+F_{n-2}$ for $n>2$.
\begin{enumerate}[(a)]
\item Prove that $\text{gcd}(F_n,F_{n+1})=1$.
  (Consecutive Fibonacci numbers are relatively prime.)
\item Prove that for $n\ge 1$, $F_m|F_{mn}$.
\end{enumerate}

\vspace{0.1in}

\item \textbf{*Problem 10.22(a,d).}
You may find Bezout's identity useful for answering these questions.
\begin{enumerate}[(a)]
\item Prove that consecutive integers $n$ and $n+1$ are relatively prime.
\setcounter{enumi}{3}
\item For $k\in\Z$, prove that $2k+1$ and $9k+4$ are relatively prime.
\end{enumerate}

\vspace{0.1in}

\item \textbf{Problem 10.22(b-c,e).}
You may find Bezout's identity useful for answering these questions.
\begin{enumerate}[(a)]
\setcounter{enumi}{1}
\item For which positive $n$ are the pair $n$ and $n+2$ relatively prime?
  Prove your answer.
\item Let $p$ be a prime.
  For which positive $n$ are the pair $n$ and $n+p$ relatively prime?
  Prove your answer.
  [Hint: If $n$ is not a multiple of $p$ then $\text{gcd}(n,p)=1$.]
\setcounter{enumi}{4}
\item As a function of $k\in\Z$, compute $\text{gcd}(2k-1,9k+4)$.
\end{enumerate}

\vspace{0.1in}

\item \textbf{Problem 10.29.}
Solve each measuring problem, or explain why it can't be done.
(You have unlimited water.)
\begin{enumerate}[(a)]
\item Using 6- and 15-gallon jugs, measure (i)~3~gallons (ii)~4~gallons (iii)~5~gallons.
\item Using 5- and 11-gallon jugs, measure (i)~6~gallons (ii)~7~gallons.
\end{enumerate}

\vspace{0.1in}

\item \textbf{*Problem 10.32.}
For $k\in\N$, show that $2^k-1$ and $2^k+1$ are relatively prime.

\vspace{0.1in}

\item \textbf{Problem 11.3.}
Give the degree sequences of $K_{n+1}$, $K_{n,n}$, $L_n$, $C_n$, $S_{n+1}$, and $W_{n+1}$.

\vspace{0.1in}

\item \textbf{Problem 11.5.}
A graph is regular if every vertex has the same degree.
Which of these graphs are regular?
\begin{enumerate}[(a)]
\begin{multicols}{7}
\item $K_6$
\item $K_{4,5}$
\item $K_{5,5}$
\item $L_6$
\item $S_6$
\item $W_4$
\item $W_5$
\end{multicols}
\end{enumerate}

\vspace{0.1in}

\item \textbf{Problem 11.10.}
Give graphs with these degree distributions, or explain why you can't.
Verify $2|E|=\displaystyle\sum_{i=1}^n \delta_i$.
\begin{enumerate}[(a)]
\begin{multicols}{4}
\item $[5,3,3,2,1]$
\item $[3,2,1,1,1]$
\item $[3,3,2,1]$
\item $[3,3,3,3,3]$
\item $[3,3,3,3,3,3]$
\item $[3,3,2,2,2]$
\item $[4,4,4,4,4]$
\item $[4,4,3,2,1]$
\item $[4,3,3,2,2]$
\item $[3,3,3,2,2]$
\item $[3,3,3,3,2]$
\item $[5,3,2,2,2]$
\end{multicols}
\end{enumerate}

\vspace{0.1in}

\item \textbf{*Problem 11.11.}
In a graph only the two vertices $u,v$ have odd degree.
Prove there is a path from $u$ to $v$.

\vspace{0.1in}

\item \textbf{Problem 11.13.}
Compute the number of edges in the following graphs:
\begin{enumerate}[(a)]
\begin{multicols}{7}
\item $K_n$
\item $K_{n,\ell}$
\item $W_n$
\end{multicols}
\end{enumerate}

\newpage

\item \textbf{Problem 11.15.}
A graph is $r$-regular if every vertex has the same degree~$r$. Show:
\begin{enumerate}[(a)]
\item If $r$ is even and $n>r$, there is an $r$-regular graph with $n$ vertices. (Tinker!)
\item If $r$ is odd and $n$ is odd, there is no $r$-regular graph with $n$ vertices.
\item If $r$ is odd and $n>r$ is even, there is an $r$-regular graph with $n$ vertices.
\item An $r$-regular graph with $4k$ vertices must have an even number of edges.
\end{enumerate}

\vspace{0.1in}

\item \textbf{*Problem 11.17.}
A graph $G$ has $n$ vertices.
\begin{enumerate}[(a)]
\item What is the maximum number of edges $G$ can have and not be connected? Prove it.
\item What is the minimum number of edges $G$ can have and be connected? Prove it.
\end{enumerate}

\vspace{0.1in}

\item \textbf{*Problem 11.27.}
Every vertex degree in a graph is at least 2.
Prove that there is at least one cycle.

\end{itemize}

\end{document}
