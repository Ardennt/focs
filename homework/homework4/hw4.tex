\documentclass{article}
\usepackage[margin=1.25in]{geometry}
\usepackage{amsmath, amssymb, setspace, enumerate, enumitem}

\begin{document}
  \begin{enumerate}[label=\textbf{10.32}]
    \item for $k \in \mathbb{Z}$, show that $2^k - 1$ and $2^k + 1$ are relatively prime.
    
    \begin{itemize}
      \item If two num are relatively prime, then $gcd(a,b) = 1$
      \item By definition, we know $K \in \mathbb{N}$, making $2^k-1$ and $2^k+1 =1$, $3$ respectively when $k=1$. We state that $gcd(a,b)$ is also in $\mathbb{N}$. Which means $d \in \mathbb{N}$.
      \item We know that $gcd(2^k-1, 2^k+1) = d$, now we must prove $d=1$ by direct proof.
    \end{itemize}

    $d|2^k-1$ tells us $2^k-1 = pd$ for $p\in\mathbb{Z}$\\
    $d|2^k+1$ tells us $2^k+1 = qd$ for $q\in\mathbb{Z}$\\
    rearranging $2^k - 1 = pd$, we can get $2^k = pd + 1$\\
    plugging in, we get $pd + 1 + 1 = qd$\\
    moving everything around, we get $2 = d(q-p)$.\\
    This tells us that, to satisfy this equation, we only have 2 possibilities of what d can be $\rightarrow d= 1$, $d=2$ since $d\in\mathbb{N}$\\[0.25in]
    We can show that $d\neq 2$ by contradiction:\\
    We know that $2^k$ is an even number, we can write $2^k = 2i$ for $i \in \mathbb{N}$\\
    Both numbers $2^k - 1$ and $2^k+1$ are odd numbers, we can rewrite it as $2^k + 1 = 2i + 1$ and $2^k - 1 = 2i-1$\\
    To prove $gcd(2i + 1, 2i - 1) \neq 2$, we contradict it and say $gcd(2i+1, 2i-1)=2$\\
    We can say $2|2i+1$ and $2|2i-1$, which means that $2i + 1 = 2z$ and $2i - 1 = 2h$, respective, where $z,h\in \mathbb{Z}$ that satisfy the equation\\
    \begin{tabular}{c | c}
      $1 = 2z-2i$ & $-1 = 2h - 2i$\\
      $1 = 2(z-i)$ & $-1 = 2(h-i)$\\
      $\frac{1}{2} = (z-i)$ & $\frac{-1}{2} = (h-i)$
    \end{tabular}\\
    Oops, the sum of the two integers can never result in a fraction! We have derived a contradiction, proving that $d \neq 2$\\
    This leaves us with $d=1$, proving that $gcd(2^k-1,2^k+1) = d$, where $d=1$. Showing that these two numbers are relatively prime.
  \end{enumerate}

  \begin{enumerate}[label=\textbf{11.17}]
    \item A graph $G$ has $n$ vertices
    \begin{enumerate}
      \item What is the maximum number of edges $G$ can have and not be connected? Prove it.\\[0.25in]
      Consider a graph G with multiple connected components. Let's call the different connected components $G^\prime$ and $G^{\prime\prime}$ and so on... Each of the connected components are disconnected from each other. In short, $G$ is a connected graph with multiple connected components.\\[0.25in]
      We prove that there exist a maximum number of edges a graph may have by deriving first a formula.\\[0.25in]
      Assume a connected components $G^\prime$ in $G$. We can say the number of vertices in this connected component is $n^\prime$. Considering the question is asking for the maximum number of edges possible, we can consider back to the Handshaking Theorem, and introduce a unique handshake to each of the vertices with all other vertices. With this, we create a complete graph of $n^\prime$ vertices.\\[0.25in]
      Now we know that for a connected component to contain the maximum number of edges possible, it must be a complete graph.\\[0.25in]
      We prove now the number of vertices per component that yield the highest number of edges overall.\\[0.25in]
      In $G^\prime$, we can have at most $n^\prime - 1$ degrees per vertex, assuming $n^\prime \geq 1$. Each vertex can be connected to every other vertex other than itself, then we derive $n^\prime (n^\prime - 1)$ for all possible edged. We exclude the duplicates by dividing by 2, leaving us with $e^\prime = \frac{n^\prime ( n^\prime - 1)}{2}$\\[0.25in]
      In $G^{\prime\prime}$, we can use the same logic to derive $e^{\prime\prime} = \frac{n{\prime\prime}(n{\prime\prime} - 1)}{2}$\\[0.25in]
      The maximum number of edges we can have in a disconnected graph $G$ is $e = \frac{(n-1)(n-2)}{2}$\\[0.25in]
      Therefore, the following must be true:
      \begin{align*}
        e^\prime + e^{\prime\prime} &= e\\
        \frac{n^\prime (n^\prime - 1)}{2} + \frac{n^{\prime\prime}(n^{\prime\prime}-1)}{2} &= \frac{(n-1)(n-2)}{2}\\
      \end{align*}
      \begin{center}
        We can simplify $n^{\prime\prime} = n - n^\prime$, where
      \end{center}
      \begin{align*}
        \frac{n^\prime(n^\prime-1)}{2} + \frac{(n-n^\prime)(n-n^\prime-1)}{2} &= \frac{(n-1)(n-2)}{2}\\
      \end{align*}
      \begin{center}
        expand both sides and simplify
      \end{center}
      \begin{align*}
        \frac{2(n^\prime)^2 + n^2 - 2n^\prime n - n}{2} = \frac{n^2 - 2n - n + 2}{2}
      \end{align*}
      \begin{center}
        discard redundant components, leaving us with
      \end{center}
      \begin{align*}
        -2n^\prime n + 2(n^\prime)^2 &= -2n + 2\\
        2(-n^\prime n + (n^\prime)^2) &= 2(-n+1)\\
        -n^\prime n + (n^\prime)^2 &= -n + 1\\
      \end{align*}
      \begin{center}
        We must find values $n^\prime$ that satisfy this statement
      \end{center}
      \begin{align*}
        -n^\prime n + (n^\prime)^2 + n - 1 &= 0\\
        (n^\prime)^2 - n^\prime n + n^\prime - n^\prime + n - 1 &= 0\\
        (n^\prime - 1)(n^\prime - n + 1) &= 0\\[0.25in]
      \end{align*}
      The only values of $n^\prime$ that achieves this equality are $n^\prime = 1$ and $n^\prime = n - 1$, proving that for $G$ to contain the maximum number of edges but remain disconnected. It must contain 2 different components where one must contain 1 vertex and other must contain $n-1$ vertices $\hfill \blacksquare$
    \item What is the minimum number of edges $G$ can have and be connected? Prove it.\\[0.25in]
    For a connected graph $G$ to contain a minimum number of edges between vertices n, it must be interpreted as a single line where each of the endpoints connect to only one single vertex. The formula is $e = n - 1$, for $e$ the minimum edges for $n$ vertices.\\[0.25in]
    We prove by contradition that there exist no smaller number of edges for $n$ number of vertices\\[0.25in]
    Assume $e = n-k$, where $k \in \mathbb{Z}$ and $k > 1$\\
    Assume $G$ is a graph where its endpoints are connected to one other vertex\\[0.25in]
    A connected graph must satisfy the following condition: you may traverse from vertex $u$ to $v$ for any $u$, $v \in G$.\\
    Assume vertex $u$, $v$, $z \in G$, where $v$ is connected to both $u$ and $v$. An increase in $k$ removes one of the edges in $G$ such that $u$, $v \in G^\prime$ and $z \in G^{\prime\prime}$. By our definition of a conncted graph, now there exist no path from $z$ to $v$ or $z$ to $u$, rendering it a disconnected graph.\\[0.25in]
    We prove by contradiction that there exists no connceted graph with less edges $e$ than $n-1$, or $e = n-1$ $\hfill\blacksquare$
    \end{enumerate}
  \end{enumerate}
\end{document}