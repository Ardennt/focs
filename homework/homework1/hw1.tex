\documentclass[11pt]{article}
\usepackage{datetime}
\usepackage{color,array,graphics}
\usepackage{enumerate}
\usepackage[pdftex, colorlinks, linkcolor=red,citecolor=red,urlcolor=blue]{hyperref}
\usepackage{ulem}

\setlength{\parindent}{0cm}

\setlength{\parskip}{0.3cm plus4mm minus3mm}

\textwidth  6.5in
\oddsidemargin +0.0in
\evensidemargin +0.0in
\textheight 9.0in
\topmargin -0.5in

\usepackage{upquote,textcomp}
\usepackage{amssymb,amsmath,amsfonts,amsthm}
\usepackage{graphicx}
\usepackage{multicol}
\usepackage[T1]{fontenc}

\def\OR{\vee}
\def\AND{\wedge}
\def\imp{\rightarrow}

\DeclareSymbolFont{AMSb}{U}{msb}{m}{n}
\DeclareMathSymbol{\N}{\mathbin}{AMSb}{"4E}
\DeclareMathSymbol{\Z}{\mathbin}{AMSb}{"5A}
\DeclareMathSymbol{\R}{\mathbin}{AMSb}{"52}
\DeclareMathSymbol{\Q}{\mathbin}{AMSb}{"51}
\DeclareMathSymbol{\I}{\mathbin}{AMSb}{"49}
\DeclareMathSymbol{\C}{\mathbin}{AMSb}{"43}

\begin{document}
\thispagestyle{empty}   %% skips page number on the first page

\begin{center}
\large
\textbf{CSCI 2200 --- Foundations of Computer Science (FoCS) \\
Homework 1 (document version 1.0)}
\end{center}

\section*{Overview}
\begin{itemize}
\item This homework is due by 11:59PM on Thursday, September~15
\item You may work on this homework in a group of no more than four students;
  unlike recitation problem sets,
  \textbf{your teammates may be in any section}
\item You may use at most \textbf{two} late days on this assignment
\item Please start this homework early and ask questions during
  office hours and at your September~14 recitation section;
  also ask (and answer) questions on the Discussion Forum 
\item Please be concise in your answers;
  even if your solution is correct, if it is not well-presented,
  you may still lose points
\item You can type or hand-write (or both) your solutions
  to the required graded problems below;
  \textbf{all work must be organized in one PDF that lists
  all teammate names}
\item You are strongly encouraged to use LaTeX, in particular for
  mathematical symbols;
  see references in Course Materials
\item \textbf{EARNING LATE DAYS:}
  for each homework that you complete using LaTeX
  (including any tables, graphs, etc., i.e.,~no hand-written anything),
  you earn one additional late day;
  you can draw graphs and other diagrams
  in another application and include them as image files
\end{itemize}

\vspace{0.2in}

\begin{center}
\includegraphics[scale=0.5]{math-bitmoji.png}
\end{center}

\newpage
\section*{Warm-up exercises}
The problems below are good practice problems to work on.
Do not submit these as part of your homework submission.
\textbf{These are ungraded problems.}

\begin{multicols}{2}
\begin{itemize}

\item \textbf{Problem 1.26}

\item \textbf{Problem 2.19}

\item \textbf{Problem 3.4}

\item \textbf{Problem 3.13}

\item \textbf{Problem 3.14}

\item \textbf{Problem 3.22}

\item \textbf{Problem 3.24}

\item \textbf{Problem 3.43}

\item \textbf{Problem 3.47}

\end{itemize}
\end{multicols}

\section*{Graded problems}
The problems below are required and will be graded.
\begin{itemize}

\item \textbf{Problem 2.16 (Cartesian Product).}

\item \textbf{Problem 2.29}

\item \textbf{Problem 3.20 (DNF). Parts (a) and (b) only.}

\item \textbf{Problem 3.23}

\item \textbf{Problem 3.31}

\item \textbf{Problem 3.44}

\item \textbf{Problem 3.56}

\item \textbf{Problem 4.7. Part (a) only.}

\end{itemize}


All of the above problems (both graded an ungraded)
are transcribed in the pages that follow.

Graded problems are noted with an asterisk~(*).

If any typos exist below, please use the textbook description.

\newpage
\begin{itemize}

\item \textbf{Problem 1.26.}
  Two players alternately pick numbers without replacement
  from the set $\{1,2,3,\ldots,9\}$.
  The first player to obtain three numbers that sum to 15 wins.
  What is your strategy?

\vspace{0.1in}

\item \textbf{*Problem 2.16 (Cartesian Product).}
  Let $A=\{1,2,3\}$ and $B=\{a,b,c,d\}$.
  The Cartesian product $A\times B$ is the set of pairs formed
  from elements of A and elements of B,
  $$A\times B = \{(a,b)\ |\ a\in A,\ b\in B\}$$

  \begin{enumerate}[(a)]
  \item List the elements in $A\times B$.
    What is $|A\times B|$?
  \item List the elements in $B\times A$.
    What is $|B\times A|$?
  \item List the elements in $A\times A=A^2$.
    What is $|A\times A|$?
  \item List the elements in $B\times B=B^2$.
    What is $|B\times B|$?
  \end{enumerate}

  Generalize the definition of $A\times B$ to a Cartesian product
  of three sets $A\times B\times C$.

\vspace{0.1in}

\item \textbf{Problem 2.19.}
  How many binary sequences are of length $1,2,3,4,5$?
  Guess the pattern.

\vspace{0.1in}

\item \textbf{*Problem 2.29.}
  Mimic the method we used to prove $\sqrt{2}$ is irrational and
  prove $\sqrt{3}$ is irrational.
  Now use the same method to try and prove $\sqrt{9}$ is irrational.
  What goes wrong?

\vspace{0.1in}

\item \textbf{Problem 3.4.}
  Define the propositions $p=$~``Kilam is a CS major''
  and $q=$~``Kilam is a hockey player''.
  Use the connectors $\OR$, $\AND$, $\imp$ to formulate these claims.
  \begin{enumerate}[(a)]
  \item Kilam is a hockey player and CS major.
  \item Kilam either plays hockey or is a CS major.
  \item Kilam plays hockey, but he is not a CS major.
  \item Kilam is neither a hockey player nor a CS major.
  \item Kilam is a CS major or a hockey player, not both.
  \item Kilam is not a hockey player but is a CS major.
  \end{enumerate}

\vspace{0.1in}

\item \textbf{Problem 3.13.}
  If it rains on a day, it rains the next day.
  Today it didn't rain.
  On which days must there be no rain? \\
  (a) Tomorrow.
  (b) All future days.
  (c) Yesterday.
  (d) All previous days.

\vspace{0.1in}

\item \textbf{Problem 3.14.}
  For $p=$~``You're sick'', $q=$~``You miss the final'',
  $r=$~``You pass FOCS'', translate into English:
  \begin{enumerate}[(a)]
  \item $q\imp\neg r$.
  \item $(p\imp\neg r)\OR(q\imp\neg r)$.
  \item $(p\AND q)\OR(\neg q\AND r)$.
  \end{enumerate}

\newpage

\item \textbf{*Problem 3.20 (DNF). Parts (a) and (b) only.}
  Use $\neg$, $\AND$, $\OR$ to give compound propositions with these truth-tables.

  [Hint: You need only consider the rows which are \textsc{t}
  and use \textsc{or} of \textsc{and}'s.]
  \begin{multicols}{2}
  \begin{enumerate}[(a)]
  \item
    \begin{tabular}{cc|c}
      $q$&$r$&\\
      \hline
      \textsc{t}&\textsc{t}&\textsc{f}\\
      \textsc{t}&\textsc{f}&\textsc{t}\\
      \textsc{f}&\textsc{t}&\textsc{f}\\
      \textsc{f}&\textsc{f}&\textsc{f}\\
    \end{tabular}
  \item
    \begin{tabular}{cc|c}
      $q$&$r$&\\
      \hline
      \textsc{t}&\textsc{t}&\textsc{f}\\
      \textsc{t}&\textsc{f}&\textsc{t}\\
      \textsc{f}&\textsc{t}&\textsc{f}\\
      \textsc{f}&\textsc{f}&\textsc{t}\\
    \end{tabular}
  \end{enumerate}
  \end{multicols}

  (\textsc{and-or-not} formulas use only $\neg$, $\AND$, $\OR$.
   Any truth-table can be realized by an \textsc{and-or-not} formula.
   Even more, one can construct an \textsc{or} or \textsc{and}'s,
   the \textit{disjunctive normal form (DNF)}.)

\vspace{0.1in}

\item \textbf{Problem 3.22.}
  How many rows are in the truth table of $\neg(p\OR q)\AND\neg r$?
  Give the truth table.

\vspace{0.1in}

\item \textbf{*Problem 3.23.}
  \begin{enumerate}[(a)]
  \item Give the truth-table for these compound propositions.
    $$p\AND\neg p;\ \ \ p\OR\neg p;\ \ \ p\imp(p\OR q);\ \ \ ((p\imp q)\AND(\neg q))\imp\neg p$$

  \item How many rows are in the truth-table of the proposition
    $(p\OR q)\imp(r\imp s)$?

  \item Show that $(p\imp q)\OR p$ is ALWAYS true.
    This is called a tautology.
  \end{enumerate}

\vspace{0.1in}

\item \textbf{Problem 3.24.}
  Let $q\imp p$ be \textsc{f} and $q\imp r$ be \textsc{t}.
  Answer \textsc{t}/\textsc{f}: (a)~$p\OR q$ (b)~$p\imp q$ (c)~$p\AND q\AND r$.

\vspace{0.1in}

\item \textbf{*Problem 3.31.}
  Use truth tables to determine the logical equivalence of the compound statements.
  \begin{multicols}{2}
  \begin{enumerate}[(a)]
  \item $(p\imp q)\imp r$ and $p\imp (q\imp r)$
  \item $(p\AND\neg q)\OR q$ and $p\OR q$
  \end{enumerate}
  \end{multicols}

\vspace{0.1in}

\item \textbf{Problem 3.43.}
  For $x\in\{1,2,3,4,5\}$ and $y\in\{1,2,3\}$,
  determine \textsc{t}/\textsc{f} with short justifications.
  \begin{enumerate}[(a)]
  \item $\exists x:x+3=10$
  \item $\forall y:y+3\le7$
  \item $\exists x:(\forall y:x^2<y+1)$
  \item $\forall x:(\exists y:x^2+y^2<12)$
  \end{enumerate}

\vspace{0.1in}

\item \textbf{*Problem 3.44.}
  For $x,y\in\Z$, determine \textsc{t}/\textsc{f} with short justifications.
  \begin{enumerate}[(a)]
  \item $\forall x:(\exists y:x=5/y)$
  \item $\forall x:(\exists y:y^4-x<16)$
  \item $\forall x:(\exists y:\log_2 x\ne y^3)$
  \end{enumerate}

\vspace{0.1in}

\item \textbf{Problem 3.47.}
  Use quantifiers to precisely formulate the associative laws
  for multiplication and addition and the distributive law
  for multiplication over addition.

\vspace{0.1in}

\item \textbf{*Problem 3.56.}
  In which (if any) of the domains $\N,\Z,\Q,\R$ are these claims \textsc{t}?
  ($x$ and $y$ can have different domains.)
  \begin{enumerate}[(a)]
  \item $\exists x:x^2=4$
  \item $\exists x:x^2=2$
  \item $\forall x: (\exists y:x^2=y)$
  \item $\forall y: (\exists x:x^2=y)$
  \end{enumerate}

\vspace{0.1in}

\item \textbf{*Problem 4.7. Part (a) only.}
  Give direct proofs:
  \begin{enumerate}[(a)]
  \item $x,y\in\Q\imp xy\in\Q$.
  \end{enumerate}

\end{itemize}

\end{document}
